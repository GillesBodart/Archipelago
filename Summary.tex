% format de la feuille
% taille police globale (pas les titres)
% recto (oneside) ou recto-verso (twoside)
% fleqn pour numéroter les équations à droite
% classe : article, book, report
\documentclass[a4paper,12pt,twoside, fleqn]{report}

% ---------------------------------------------------------------------------------------------------------
% LANGUES ET POLICE
\usepackage[frenchb, english]{babel} % langue principale : français
\usepackage[utf8]{inputenc} % encodage
\usepackage[T1]{fontenc} % accents
\usepackage{lmodern} % police vectorielle
\usepackage{ulem} % souligner
\usepackage{color} % pour écrire en couleur

% ---------------------------------------------------------------------------------------------------------
% CARACTERES SPECIAUX + MATH
\usepackage{csquotes} % beaux guillemets \enquote{texte}
\usepackage{amssymb} % symboles mathématiques
\usepackage{listings} % inclure du joli code
\usepackage {amsmath} % formules mathématiques, flèches etc
\usepackage{amsfonts} % fraktur etc. 
% ---------------------------------------------------------------------------------------------------------
% LISTES
\usepackage{enumitem} % customiser les listes
%\frenchbsetup{StandardLists=true} % éviter les conflits avec enumitem
\renewcommand{\labelitemi}{\textbullet} % un rond à la place du tiret dans les listes
\setlist[itemize]{noitemsep, topsep=0pt}
\setlist[enumerate]{noitemsep, topsep=0pt}
\setlist[description]{noitemsep, topsep=0pt}
% ---------------------------------------------------------------------------------------------------------
% TABLEAUX ET FIGURES
\usepackage[sc,footnotesize]{caption} % légendes des tableaux et figures
\usepackage{subfig} % plusieurs figures côte à côte (subfloat)
\usepackage{longtable} % autoriser que les longs tableaux débordent sur plusieurs pages
\usepackage{multicol} % fusionner les colonnes dans un tableau
\usepackage{multirow} % fusionner les lignes dans un tableau
\usepackage{caption} % pour les légendes
\usepackage{slashbox} % pour les Tableaux comparatifs
\usepackage{graphicx} % insérer image
\graphicspath{{figures/}} % dossier dans lequel sont les images
%\usepackage{tikz} 	% outil de modélisation de formes

% ---------------------------------------------------------------------------------------------------------
% MISE EN PAGE 
\usepackage{lscape}
\usepackage[nodisplayskipstretch]{setspace}

\raggedbottom

% - Marges
\usepackage[top=2cm, bottom=2cm, left=2cm, right=2cm ]{geometry} % marges

% - Page style
\pagestyle{headings}

% - Alinéas et espacements entre paragraphes
\usepackage{parskip}
\parskip = 10pt % espace
\parindent = 0pt % alinéa

% - Style des sections
\usepackage{titlesec}
\titleformat*{\section}{\Large\scshape}
\titleformat*{\subsection}{\large\scshape}
\titleformat*{\subsubsection}{\scshape}

\pagestyle{plain}
% ---------------------------------------------------------------------------------------------------------
  
\author{Gilles Bodart}
\date{\today}
\title{%
    \begin{minipage}\linewidth
        \centering\bfseries\sffamily
        Archipelago : Un framework de peristence de graph de données.
        \vskip4pt
        \large https://github.com/GillesBodart/Archipelago
    \end{minipage}
}
\begin{document}
\maketitle
 	
\chapter{Introduction}
\section{historique} 
Un historique sur le pourquoi et le comment les bases de données de types NoSql et plus précisément les BD\footnote{Base de Donnée} orientées graphes ont vu le jour.
\section{State of the art}	
Le premier objectif de ce mémoire est d'établir un état de l'art sur les différents types de BD. Nous recencerons dans cette section, les différentes BDOG\footnote{Base de Donné Orientée Graphe}. présentes sur le marché à l'heure actuelle :
\begin{itemize}
\item Neo4J
\item OrientDb
\item Blazegraph
\item InfiniteGraph
\item Infogrid
\item ArangoDb
\end{itemize}
Une explication de haut niveau sera faite sur les différence ainsi que sur les points communs de ces dernières. Chaque élément fera le sujet d'une analyse plus détaillée dans le chapitre "Analyse Technique".


\chapter{Analyse technique}
\section{Application possibles des BDOG}
Cette section se concentrera sur l'analyse des besoins utilisateurs, il tentera de répondre aux questions suivantes: 
\begin{itemize}
\item Pourquoi utiliser une BDOG plutot qu'une BD relationnelle comme Oracle ou MySql ?
\item On parle de Base de donnée orientée graph mais quelle est la différence entre un lien entre deux noeud et une relation entre deux table ?
\item Si nous devions choisir un exemple qui nécessiterait l'utilisation d'une BDOG, quel serait il ?
\end{itemize}
\section{Critères de comparaisons}
L'établissement d'une liste non exaustive de critères de comparaisons objectifs sera établie. Elle me permettra de comparer les différentes BDOG. Les possibilités de réponse à ces critère seront, dans les limites du possible rémmenée au choix dual, Oui/Non. Cela permettra d'établir un arbre de décision binaire sur base de besoins clairs.\\TODO Dessin arbre binaire

\section{Comparaison des plus grandes BDOG} 
\begin{center}
\begin{tabular}[c]{|l|c|c|c|}
\hline
\backslashbox {Critère}{Bases de données} & Neo4J & OrientDB & ArangoDB  \\
\hline
Critère 1 & \checkmark & & \checkmark \\
\hline
Critère 2 & & \checkmark & \\
\hline
Critère 3 & \checkmark & & \\
\hline
\end{tabular}
\end{center}
Une idée d'arbre de décision devrait, à la fin de cette section, permettre à un utilisateur muni de ses besoins, de choisir la BDOG la plus adaptée à son projet. 
\chapter{Le framework}
\section{Utilisation}
Dans l'état actuele de l'avancement de ce mémoire, deux pistes sont envisagée:
\begin{itemize}
\item Création d'une API qui sera utilisée par l'application dans le but de simplifier les différentes oppérations sur la base de donnée. Exemple Hibernate pour JDBC.\\
TODO Schema
\item Création d'un systême d'abstraction qui va englober l'utilisation et de ce fait cacher l'implémentations des différentes oppérations.\\
TODO Schema 
\end{itemize}
\section{Schema conceptuel}
Les schémas conceptuels représentant une modèle exemple annoté des éléments du framework sera fourni et commenté dans cette section. Cela permettra au lecteur une meilleur compréhension.
\section{Documentation}
Une documentation claire et précise sur l'utilisation du framework Archipelago sera présente dans cette section. Un ensemble entre une documentation fonctionnelle et une documentation technique faite avec JavaDoc.
\section{Processus}
Description du processus implémenté sur base d'un exemple claire. Explications des différents choix d'implémentations et de chaque étape.
\section{Points forts}
Autocritique du framework, sur base de test qualitatif et ou quantitatif. Evaluations : usability, performence, qualité, cohérence
\section{Points faibles}
Autocritique du framework, sur base de test qualitatif et ou quantitatif.
Evaluations : usability, performence, qualité, cohérence
\section{Retour d'information}
Si le temps nous le permet, une analyse des retours utilisateur sera faite en fin de mémoire.
\chapter{Conclusion}
\section{Piste de réflexions}
Une introspection sur le projet sera expliqué dans cette section, les idées innachevés y seront décrites en tant que piste de réflexions.
\section{Archipelago en résumé}
Le mémoire sera conclu avec un explication transversale et complète du framework, permettant au lecteur de garder une bonne impression sur le nouvl outil que sera ce framework.
\end{document}