% format de la feuille
% taille police globale (pas les titres)
% recto (oneside) ou recto-verso (twoside)
% fleqn pour numéroter les équations à droite
% classe : article, book, report
\documentclass[a4paper,12pt,twoside, fleqn]{report}

% ---------------------------------------------------------------------------------------------------------
% LANGUES ET POLICE
\usepackage[frenchb, english]{babel} % langue principale : français
\usepackage[utf8]{inputenc} % encodage 
\usepackage[T1]{fontenc} % accents
\usepackage{lmodern} % police vectorielle
\usepackage{ulem} % souligner
\usepackage{color} % pour écrire en couleur 

% ---------------------------------------------------------------------------------------------------------
% CARACTERES SPECIAUX + MATH
\usepackage{csquotes} % beaux guillemets \enquote{texte}
\usepackage{amssymb} % symboles mathématiques
\usepackage{listings} % inclure du joli code
\usepackage {amsmath} % formules mathématiques, flèches etc
\usepackage{amsfonts} % fraktur etc. 
% ---------------------------------------------------------------------------------------------------------
% LISTES
\usepackage{enumitem} % customiser les listes
%\frenchbsetup{StandardLists=true} % éviter les conflits avec enumitem
\renewcommand{\labelitemi}{\textbullet} % un rond à la place du tiret dans les listes
\setlist[itemize]{noitemsep, topsep=0pt}
\setlist[enumerate]{noitemsep, topsep=0pt}
\setlist[description]{noitemsep, topsep=0pt}
% ---------------------------------------------------------------------------------------------------------
% TABLEAUX ET FIGURES
\usepackage[sc,footnotesize]{caption} % légendes des tableaux et figures
\usepackage{subfig} % plusieurs figures côte à côte (subfloat)
\usepackage{longtable} % autoriser que les longs tableaux débordent sur plusieurs pages
\usepackage{multicol} % fusionner les colonnes dans un tableau
\usepackage{multirow} % fusionner les lignes dans un tableau
\usepackage{caption} % pour les légendes
\usepackage{slashbox} % pour les Tableaux comparatifs
\usepackage{graphicx} % insérer image
\graphicspath{{figures/}} % dossier dans lequel sont les images
%\usepackage{tikz} 	% outil de modélisation de formes

% ---------------------------------------------------------------------------------------------------------
% MISE EN PAGE 
\usepackage{lscape}
\usepackage[nodisplayskipstretch]{setspace}

\raggedbottom

% - Marges
\usepackage[top=2cm, bottom=2cm, left=2cm, right=2cm ]{geometry} % marges

% - Page style
\pagestyle{headings}

% - Alinéas et espacements entre paragraphes
\usepackage{parskip}
\parskip = 10pt % espace
\parindent = 0pt % alinéa

% - Style des sections
\usepackage{titlesec}
\titleformat*{\section}{\Large\scshape}
\titleformat*{\subsection}{\large\scshape}
\titleformat*{\subsubsection}{\scshape}

\pagestyle{plain}
% ---------------------------------------------------------------------------------------------------------
\selectlanguage{frenchb}
\author{Gilles Bodart}
\date{\today}
\title{%
    \begin{minipage}\linewidth
        \centering\bfseries\sffamily
        Archipelago : Un framework de peristence de graph de données.
        \vskip4pt
        \large https://github.com/GillesBodart/Archipelago
    \end{minipage}
}
\begin{document}
\maketitle
\tableofcontents
\part{State of the art}
\chapter{Introduction}

\section{Historique}  

Un SGBD\footnote{Système de gestion de base de données} est par définition un ensemble de procédés permettant d'organiser et de stocker des informations (potentiellement de gros volumes). Si stocker et retrouver l'information est un des plus grand challenge d'un SGBD, une communauté de développeur, pensent que ces système devraient pouvoir offrir d'autres fonctionnalités. 

A partir des années 1980, le modèle relationnelle supplante les autres formes de structures de donnée.

Les évolutions logicielles suivant naturellement les évolutions matérielles, la généralisation des interconnexion des réseaux, l'augmentation de la bande passante, la diminution du cout des machines, la miniaturisation des espaces de stockage, ... de nouvelles opportunités sont arrivé au XXI\up{e} siècle.

Les entreprises comme Google, Amazon, Facebook, Twitter, ... sont tour à tour arrivés aux limites du modèle Relationnel. Que ce soit a cause de volumes astronomiques (plus de 100 pétaoctets ) ou du nombre de requêtes par secondes, il fallut développer une nouvelle façon de gérer les données.

Le NoSql\footnote{Not Only Sql} découle de ce genre de problèmes, ces modèles arrivent avec des approche optimisée pour des secteurs spécifiques. \\
Comme les modèles NoSql représentent ce qui n'est pas Relationnel, par soucis de classification, nous allons distinguer 4 usages principaux :

\begin{itemize}
\item Performances : L'objectif du SGBD sera d'augmenter au maximum les performances de la manipulation des données. 
\item Structures simples : Pour s'affranchir de la rigidité du modèle relationnel, la structure sera généralement simplifiée, en utilisant une représentation plus souple comme le JSON par exemple.
\item Structures spécifiques : Certain moteur NoSql sont liés a des besoins spécifiques, la structure de représentation de donnée sera dès lors focalisée sur un cas d'utilisation.
\item Volumétries : Un des principal aspect important des SGBD NoSql est leur capacité de gérer la montée en charge de données. La distribution des traitements au travers de plusieurs clusters est un facteur très important dans la pluspart des applications BigData.
\end{itemize} 

Et nous allons aussi distinguer 4 grandes familles de représentation de Schéma de données :

\begin{itemize}
\item Document : L'utilisation de format spécifiques tels que le très rependu JSON permet de stocker les données sur base de fichier.
\item Clé / Valeur : Le système le plus simple, il manipule des paires de clé/valeurs, ou accède à un élément en fonction d'une table de hachage.
\item Colonne : Inspiré de Google BigTable, la structure ressemble à la table relationnelle. On peut la comparer à une table de hachage qui va référencer une ou plusieurs colonnes.
\item Graph : La famille Graph se distingue du fait que les entitées ne sont pas considéré comme des entités indépendante, mais que la relation être ces objets est tout aussi important que le contenu.
\end{itemize} 
 
 
\section{De nos jours}	
Le premier objectif de ce mémoire est d'établir un état de l'art sur les différents types de BD. Nous recenserons dans cette section, les différentes BDOG\footnote{Base de Donné Orientée Graphe}. présentes sur le marché à l'heure actuelle :
\begin{itemize}
\item Neo4J
\item OrientDb
\item Blazegraph
\item InfiniteGraph
\item Infogrid
\item ArangoDb
\end{itemize}
Une explication de haut niveau sera faite sur les différence ainsi que sur les points communs de ces dernières. Chaque élément fera le sujet d'une analyse plus détaillée dans le chapitre "Analyse Technique".


\part{Analyse technique}
\chapter{Application possibles des BDOG}
Cette section se concentrera sur l'analyse des besoins utilisateurs, il tentera de répondre aux questions suivantes: 
\begin{itemize}
\item Pourquoi utiliser une BDOG plutot qu'une BD relationnelle comme Oracle ou MySql ?
\item On parle de Base de donnée orientée graph mais quelle est la différence entre un lien entre deux noeud et une relation entre deux table ?
\item Si nous devions choisir un exemple qui nécessiterait l'utilisation d'une BDOG, quel serait il ?
\end{itemize}
\section{Critères de comparaisons}
L'établissement d'une liste non exhaustive de critères de comparaisons objectifs sera établie. Elle me permettra de comparer les différentes BDOG. Les possibilités de réponse à ces critère seront, dans les limites du possible ramenée au choix dual, Oui/Non. Cela permettra d'établir un arbre de décision binaire sur base de besoins clairs.\\TODO Dessin arbre binaire

\section{Comparaison des plus grandes BDOG} 
\begin{center}
\begin{tabular}[c]{|l|c|c|c|}
\hline
\backslashbox {Critère}{Bases de données} & Neo4J & OrientDB & ArangoDB  \\
\hline
Critère 1 & \checkmark & & \checkmark \\
\hline
Critère 2 & & \checkmark & \\
\hline
Critère 3 & \checkmark & & \\
\hline
\end{tabular}
\end{center}
Une idée d'arbre de décision devrait, à la fin de cette section, permettre à un utilisateur muni de ses besoins, de choisir la BDOG la plus adaptée à son projet. 
\chapter{Le framework}
\section{Utilisation}
Dans l'état actuele de l'avancement de ce mémoire, deux pistes sont envisagée:
\begin{itemize}
\item Création d'une API qui sera utilisée par l'application dans le but de simplifier les différentes oppérations sur la base de donnée. Exemple Hibernate pour JDBC.\\
TODO Schema
\item Création d'un systême d'abstraction qui va englober l'utilisation et de ce fait cacher l'implémentations des différentes oppérations.\\
TODO Schema 
\end{itemize}
\section{Schema conceptuel}
Les schémas conceptuels représentant une modèle exemple annoté des éléments du framework sera fourni et commenté dans cette section. Cela permettra au lecteur une meilleur compréhension.
\section{Documentation}
Une documentation claire et précise sur l'utilisation du framework Archipelago sera présente dans cette section. Un ensemble entre une documentation fonctionnelle et une documentation technique faite avec JavaDoc.
\section{Processus}
Description du processus implémenté sur base d'un exemple claire. Explications des différents choix d'implémentations et de chaque étape.
\chapter{Evaluation}
\section{Points forts}
Autocritique du framework, sur base de test qualitatif et ou quantitatif. Evaluations : usability, performence, qualité, cohérence
\section{Points faibles}
Autocritique du framework, sur base de test qualitatif et ou quantitatif.
Evaluations : usability, performence, qualité, cohérence
\section{Retour d'information}
Si le temps nous le permet, une analyse des retours utilisateur sera faite en fin de mémoire.
\chapter{Conclusion}
\section{Piste de réflexions}
Une introspection sur le projet sera expliqué dans cette section, les idées innachevés y seront décrites en tant que piste de réflexions.
\section{Archipelago en résumé}
Le mémoire sera conclu avec un explication transversale et complète du framework, permettant au lecteur de garder une bonne impression sur le nouvl outil que sera ce framework.
\part{Annexes}
\chapter{Code Sources}
\chapter{Bibliographie}
\begin{itemize}
\item https://neo4j.com/ consulté à de nombreuses reprises (Neo Technology, Inc)
\item https://www.arangodb.com/ consulté à de nombreuses reprises (ArangoDB GmbH)
\item https://orientdb.com/ consulté à de nombreuses reprises (OrientDB LTD)
\item https://snap.stanford.edu/data/
\item https://networkx.github.io/
\item http://igraph.org/redirect.html
\item https://snap.stanford.edu/data/egonets-Facebook.html
\item http://konect.uni-koblenz.de/
\item https://icon.colorado.edu/\#!/networks
\item https://neonx.readthedocs.io/en/latest/
\item J. McAuley and J. Leskovec. Learning to Discover Social Circles in Ego Networks. NIPS, 2012.
\item J. Leskovec, K. Lang, A. Dasgupta, M. Mahoney. Community Structure in Large Networks: Natural Cluster Sizes and the Absence of Large Well-Defined Clusters. Internet Mathematics 6(1) 29--123, 2009.
\item https://www.infoq.com/fr/articles/graph-nosql-neo4j
\item http://www.marginweb.com/blog/20121110/quand-et-pourquoi-utiliser-une-base-de-donnees-nosql
\item http://www.silicon.fr/base-donnees-nosql-impose-sgbdr-93305.html
\item https://prezi.com/4flswlgipwbo/nosql-not-only-sql/
\item [LIVRE] http://www.eyrolles.com/Chapitres/9782212141559/9782212141559.pdf
\end{itemize}
\end{document}